\documentclass[12pt,a4paper]{article}

\usepackage[a4paper,margin=1in]{geometry}
\usepackage[T1]{fontenc}
\usepackage[utf8]{inputenc}
\usepackage{lmodern}
\usepackage{setspace}

\usepackage{array}
\usepackage{booktabs}
\usepackage{multirow}
\usepackage{longtable}
\usepackage{siunitx}

\usepackage{graphicx}
\usepackage{float}
\usepackage{caption}
\usepackage{subcaption}

\usepackage{enumitem}
\usepackage{titlesec}

\usepackage{hyperref}

\usepackage{listings}
\usepackage{xcolor}

\title{\textbf{Electronics -- UPY22401J}}
\author{Pranauv Skandhan}
\date{}

\begin{document}

\maketitle

\tableofcontents

\newpage

\section{PN Junction Diode}

\subsection{Definition}
A PN junction diode is a two-terminal semiconductor device formed by joining p-type and n-type semiconductor materials. It allows current to flow easily in one direction and restricts flow in the opposite direction.

\subsection{Construction}
When p-type and n-type semiconductors are joined, a depletion region is formed near the junction due to recombination of electrons and holes. This region is free of mobile charge carriers and creates a built-in potential barrier.

\subsection{Biasing of PN Junction}
\begin{itemize}
    \item \textbf{Forward Bias:} P-side connected to positive terminal and n-side to negative terminal. Depletion region narrows and current flows.
    \item \textbf{Reverse Bias:} P-side connected to negative terminal and n-side to positive terminal. Depletion region widens and negligible current flows.
\end{itemize}

\subsection{V--I Characteristics}
In forward bias, current increases exponentially after the cut-in voltage. In reverse bias, a small saturation current flows until breakdown occurs.

\subsection{Applications}
\begin{itemize}
    \item Rectifiers
    \item Electronic switches
    \item Signal demodulation
\end{itemize}

\section{Zener Diode}

\subsection{Definition}
A Zener diode is a specially designed PN junction diode that operates in the reverse breakdown region to maintain a constant output voltage.

\subsection{V--I Characteristics}
The Zener diode conducts very little current in reverse bias until the breakdown voltage (Zener voltage) is reached. Beyond this point, the voltage remains nearly constant while current increases sharply.

\subsection{Zener Breakdown}
Zener breakdown occurs due to a strong electric field across the depletion region, causing electrons to tunnel through the junction.

\subsection{Zener Diode as Voltage Regulator}
A Zener diode connected in reverse bias across a load maintains a constant voltage output even when the input voltage or load current varies.

\subsection{Applications}
\begin{itemize}
    \item Voltage regulation
    \item Over-voltage protection
    \item Reference voltage circuits
\end{itemize}

\end{document}